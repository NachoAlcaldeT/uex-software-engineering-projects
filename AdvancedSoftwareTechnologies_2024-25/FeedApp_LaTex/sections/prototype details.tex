\section{Prototype Details}
\label{prototype_details}

\subsection{Prototype Description}

We created a prototype for FeedApp, an interactive app to make polls and provide them with voting and management support. Below is the detailed flow of main system functionalities:

\subsubsection{Poll Creation Flow}
\begin{itemize}
    \item If a user doesn’t have an account, they can register one, while if they do have an account, they just have to log in with said account.
    \item After logging in, users will be taken to a page where all polls are displayed and will be able to create a new poll.
    \item When users create a poll, in the poll creation interface, they have to first provide the poll title.
    \item Upon entering the title, the user is then prompted in a second interface to add voting options.
    \item Finally, when confirmed, the poll with its options is stored in the database and is displayed in the poll listing page.
\end{itemize}

\subsubsection{Voting Flow}
\begin{itemize}
    \item Guests can view the polls and vote without an account. These votes will be saved in the database with \texttt{"user\_id"} set to null.
    \item If the user has voted on a poll, the identified users have the ability to change their vote. These updates are then instantly shown in the database as well as in the system user interface.
\end{itemize}

\subsubsection{Profile and Poll Management}
\begin{itemize}
    \item This will take authenticated users to their profile, with their registered details and a list of all polls they have created.
    \item Users can edit or delete existing polls from this section, as well as change the corresponding voting options.
\end{itemize}

\subsubsection{Analytical Reports}
\begin{itemize}
    \item Implemented a NoSQL database system (\textbf{MongoDB}) to store and query analytical data. These contain statistics on registered users, polls created, and votes cast (from both identified and unidentified users).
\end{itemize}

\subsection{Data Storage Considerations}

The prototype uses a hybrid approach to data storage, which consists of using a relational and NoSQL database:


\subsubsection{Relational Database (SQLite)}

\begin{itemize}
    \item A simple-to-use, serverless, self-contained, zero-configuration, cross-platform, file-based SQL database. It is most suitable for applications that have a well-defined structure and SQL queries.
    \item Stores core operational data (users, polls, vote choices, etc.).
    \item There are related tables included in the database structure:
    \begin{itemize}
        \item \textbf{Users:} Stores basic information about users (\texttt{id}, \texttt{username}, \texttt{email}, \texttt{password}).
        \item \textbf{Polls:} Stores the polls created by users.
        \item \textbf{VoteOptions:} Lists the options available for each poll.
        \item \textbf{Votes:} Logs each vote cast, including the \texttt{user\_id} if applicable.
    \end{itemize}
\end{itemize}

\subsubsection{NoSQL Database (MongoDB)}

\begin{itemize}
    \item A non-relational (NoSQL) database management system that stores information in JSON or BSON documents, providing flexibility in the structure of the data and making it ideal for handling large amounts of unstructured or semi-structured data.
    \item \textbf{What We Used It For:} To store analytical data, like:
    \begin{itemize}
        \item Total number of registered users.
        \item Number of polls created and their relation to users.
        \item Votes cast, distinguishing between identified and unidentified users.
    \end{itemize}
    \item \textbf{MongoDB collections include:}
    \begin{itemize}
        \item \textbf{Users:} User information for analysis.
        \item \textbf{Polls:} Statistics on created polls.
        \item \textbf{Votes:} Records of votes for participation analysis.
    \end{itemize}
\end{itemize}

Using a dual design isolates operational data from analytical data, allowing for greater performance and the ability to perform specific queries.

\newpage 

